\documentclass{scrartcl}
\usepackage[utf8]{inputenc}
\usepackage[english]{babel}
\usepackage{caption}
\usepackage{listings}
\usepackage{pdfpages}
\usepackage{amsmath,amssymb}

\newcommand{\qed}{\hfill $\blacksquare$}

\lstset{frame=single,keepspaces=true,captionpos=b}

\title{Homework 02 - Probability Theory}
\author{Arne Sachtler - \textit{Registration Number: 03692662}}
\date{\today}
\subtitle{IN2064 Machine Learning}

\begin{document}
\maketitle

\section{Basic Probability} % (fold)
\label{cha:basic_probability}

\subsection{Problem 1} % (fold)
\label{sec:problem_1}

Introducing event variables as
\begin{equation}
	T: \text{person is a terrorist} \, 
\end{equation} and
\begin{equation}
	S: \text{person is assessed as terrorist}
\end{equation}
leads to a formal specification of the knowledge provided in the textual description as
\begin{equation}
	P(S|T) = 95 \% \quad \text{,} \quad P(\bar{S}|\bar{T}) = 95\% \quad \text{and} \quad P(T)=1\%. 
\end{equation}

In order to determine the probability that the person next to me actually is a terrorist, the conditional probability $P(T|S)$ is required. Applying Bayes' rule yields
\begin{eqnarray}
	P(T|S) &=& \frac{P(S|T) P(T)}{P(S)} \\
	&=& \frac{P(S|T) P(T)}{\sum_{A\in \left\{T, \bar{T}\right\}} P(A) P(S|A)}\\
	&=& \frac{95\% \cdot 1\%}{1\% \cdot 95\% + 99\% \cdot 5\%}\\
	&\approx& 16.1\%
\end{eqnarray}
Lucky me, the person next to me is a terrorist by only 16\%.

\subsection{Problem 2} % (fold)
\label{sec:problem_2}
Let the random variable $C$ be the amount of red balls placed into the box and $D$ the amount of red balls drawn from the box.
$C$ and $D$ are Binomial distributed as follows
\begin{eqnarray}
C&\sim&Bin\left(2, \frac{1}{2}\right)\\
D&\sim&Bin\left(3, \mu_r\right) \, ,	
\end{eqnarray}
where the parameter $\mu_r$ denotes the probability of drawing a red ball from the box. This parameter is dependent on the amount of red balls in the box.
Formally the required conditional probability can be expressed in terms of those random variables as $p(C=2 | D=3)$.
Again, Bayes' rule can be applied in order to compute the final conditional probability. The law of marginal probability is used and the parameter $\mu_r$ is adopted for each case.
In the following equations let $B(k|N,\mu)$ be the short form of $Bin(k|N,\mu)$.

\begin{eqnarray}
p(C=2 | D=3) &=& \frac{p(D=3 | C=2) \, p(C=2)}{\sum_{k=0}^{2} p(D=3 | C=k) \, p(C=k)}\\
&=& \frac{B(3|3,1) \, B(2|2, 0.5)}{B(3|3,0) \, B(0|2, 0.5) + B(3|3,0.5) \, B(1|2, 0.5) + B(3|3,1) \, B(2|2,0.5)}\\
&=& \frac{1 \cdot \frac{1}{4}}{0\cdot \frac{1}{4} + \frac{1}{8} \cdot \frac{1}{2} + 1 \cdot \frac{1}{4}}\\
&=& \frac{4}{5} = 80\% \, .\\
\end{eqnarray}

Under the observation of drawing three times a red ball from the box, the probability that initially two of the two balls placed in the box were red is 80\%.
\subsection{Problem 3} % (fold)
\label{sec:problem_3}
Let $N$ be the random variable modelling the trial until at first the head appears.
The probability, that a head appears after the first trial is $p$. 
Afterwards, the probability that the first head appears after the second trial is $p (1-p)$.
Therefore, the probability for the first head appearing after the n-th trial is
\begin{equation}
	p(N = n) = p (1-p)^{n-1} \, .
\end{equation}
In this case, for the fair coin, $p$ equal $\frac{1}{2}$ and $p(N = n)$ can be simplified to
\begin{equation}
	p(N = n) = \left(\frac{1}{2}\right)^{n} = \frac{1}{2^n} \, .
\end{equation}
The function evaluating the number of trials after $n$ trial is simply the identity:
\begin{equation}
	g(n) = n \, .
\end{equation}
Finally, the expected value of trial $E\left[g(n)\right]$ can be computed by
\begin{eqnarray}
	E\left[g(n)\right] &=& \sum\limits_{k=1}^{\infty}\frac{n}{2^n} = 2	
\end{eqnarray}
The expected number of heads is of course one, a each run of the experiment terminates after the first heads appeared:
\begin{equation}
	E\left[H\right] = 1 \, .
\end{equation}
Provided the first heads appeared after $n$ trials, $n-1$ tails have been observed in advance.
Therefore the expected number of tails is expressed by the following sum
\begin{eqnarray}
	E\left[T\right] &=& \sum\limits_{k=1}^\infty \frac{n-1}{2^n} \\
	&=& \sum\limits_{k=1}^{\infty}\frac{n}{2^n} - \sum\limits_{k=1}^{\infty}\frac{1}{2^n}\\
	&=& 2 - 1 = 1 \, .
\end{eqnarray}


\subsection{Problem 4} % (fold)
	\label{sec:problem_4}

For the expected value, the fact is used that $p(x) = 0$ for $x<a$ and $x>b$. First the expected value:
\begin{eqnarray}
E\left[X\right] &=& \int\limits_{-\infty}^\infty s p(s) ds = \frac{1}{b-a}\int\limits_{a}^b s ds\\
&=& \frac{1}{b-a} \left(\frac{1}{2}b^2 - \frac{1}{2}a^2\right)\\
&=& \frac{\left(b+a\right)\left(b-a\right)}{2\left(b-a\right)}\\
&=& \frac{b+a}{2} \, .
\end{eqnarray}
And for the variance:
\begin{eqnarray}
Var\left[X\right] &=& E\left[X^2\right] - E\left[X\right]^2\\
&=& \int\limits_{-\infty}^\infty s^2p(s)ds - \left(\int\limits_{-\infty}^\infty s p(s) ds\right)^2\\
&=& \frac{1}{b-a}\int\limits_a^b s^2ds - \frac{\left(a+b\right)^2}{4}\\
&=& \frac{1}{b-a}\left(\frac{1}{3}b^3 - \frac{1}{3}a^3\right) - \frac{\left(a+b\right)^2}{4}\\
&=& \frac{b^3 - a^3}{3\left(b-a\right)} - \frac{\left(a+b\right)^2}{4}\\
&=& \frac{4b^3 - 4a^3 - \left(3a^2 + 6ab + 3b^2\right)\left(b-a\right)}{12\left(b-a\right)}\\
&=& \frac{b^3 - 3ab^2 + 3a^2b - a^3}{12\left(b-a\right)}\\
&=& \frac{\left(b-a\right)^2}{12}
\end{eqnarray}
\subsection{Problem 5} % (fold)
	\label{sec:problem_5}

First the expected value. 
Using the law of marginal probability
\begin{equation}
	\int p(x|y)p(y)dy = p(x) \, ,
\end{equation}
the proof can be constructed straightforwardly
\begin{eqnarray}
	E\left[X\right] &=& E_Y\left[E_{X|Y}\left[X\right]\right]\\
	&=& \int E_{X|Y}\left[X\right]p(y)dy\\
	&=& \iint x p(x|y) p(y) \, dx\,  dy\\
	&=& \int x p(x) dx\\
	&=& E\left[X\right] \, .
\end{eqnarray}

Using the equalities
\begin{equation}
	E_Y\left[Var_{X|Y}\left[X\right]\right] = \iint x^2 p(x|y)p(y)dxdy - \int \left(\int x p(x|y)dx\right)^2p(y)dy
\end{equation}
and 
\begin{equation}
	Var_Y\left[E_{X|Y}\left[X\right]\right] = \int \left(\int x p(x|y)dx\right)^2p(y)dy - \left(\iint x(p(x|y)p(y)dx dy\right)^2
\end{equation}

the tower probability for the variance is proven
\begin{eqnarray}
& E_Y\left[Var_{X|Y}\left[X\right]\right] + Var_Y\left[E_{X|Y}\left[X\right]\right]\\ 
=& \iint x^2 p(x|y)p(y)dxdy - \int \left(\int x p(x|y)dx\right)^2p(y)dy\\
+& \int \left(\int x p(x|y)dx\right)^2p(y)dy - \left(\iint x(p(x|y)p(y)dx dy\right)^2\\
=& \iint x^2 p(x|y)p(y)dxdy - \left(\iint x(p(x|y)p(y)dx dy\right)^2\\
=& \int x^2p(x)dx - \left(\int xp(x)dx\right)^2\\
=& E\left[X^2\right] - E\left[X\right]^2\\
=& Var\left[X\right] \, .
\end{eqnarray}
\qed
	
\section{Probability Inequalities} % (fold)
\label{cha:probability_inequalities}

\subsection{Problem 6} % (fold)
\label{sec:problem_6}

To prove:
\begin{equation}
	\lim_{n\to\infty} p\left(\left|\frac{1}{n} \sum\limits_{i=1}^{n}X_i - E\left[X_i\right]\right| > \epsilon \right) = 0 \, .
\end{equation}
Using $X = \sum X_i$ the term can be bounded by Chebyshev’s inequality:
\begin{eqnarray}
		&p\left(\left|\frac{1}{n} \sum\limits_{i=1}^{n}X_i - E\left[X_i\right]\right| > \epsilon \right)\\
		=&p\left(\left|X - E\left[X\right]\right| > n\epsilon \right)\\
		\le& \frac{Var\left[X\right]}{n^2\epsilon^2} \, .
\end{eqnarray}
For $n\to\infty$ the term $\frac{Var\left[X\right]}{n^2\epsilon^2}$ goes to zero. \qed

\end{document}